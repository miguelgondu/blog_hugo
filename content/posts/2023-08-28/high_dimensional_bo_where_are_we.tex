\documentclass{article}
\usepackage[utf8]{inputenc}
\usepackage{fullpage}
\usepackage{fourier}
\usepackage{xcolor}
\usepackage[colorlinks=true, linkcolor=blue!75, citecolor=blue!75, urlcolor=blue!75]{hyperref}

\title{High-dimensional Bayesian Optimization \\ Where are we?}
\author{Miguel González Duque}
\date{}

\setlength{\parskip}{\smallskipamount}
\setlength{\parindent}{0pt}

\begin{document}

\maketitle

\section*{Introduction}

It is folk knowledge that (exact) Gaussian Process inference doesn't scale with nethier dimension nor size of the dataset. Recently, though, several efforts have been put into scaling these two aspects in both GP training and inference [TODO: cite AGW, cite Javier, cite the GPyTorch paper].

This blogpost describes these advancements and tries to answer the question \textit{where are we?} By this, we mean: how large can our datasets get (both in size and the dimension of the input data).

We will focus on high-dimensional Bayesian Optimization using GPs as surrogate models, varying both dimension and dataset size for a running example: \textit{latent space optimization} of molecules (represented as SELFIES strings) and of Super Mario Bros levels.

Some prerequisites: I assume you are familiar with Gaussian Processes and Bayesian Optimization \href{https://www.miguelgondu.com/blogposts/2023-07-31/intro-to-bo/}{to the extent I discussed them in the previous post}.

This blogpost follows, among other references listed at the end, the survey by Binois and Wycoff called \textit{A Survey on High-dimensional Gaussian Process Modeling with Application to Bayesian Optimization} \cite{BinoisWycoff:high-dimensional-bo:2022}.\footnote{\url{https://dl.acm.org/doi/pdf/10.1145/3545611}}

\section*{Running experiment: latent space optimization}

\subsection*{On SELFIES and Mario}

[Describe SELFIES strings, describe the SMB task]

[Learning latent space representations using a VAE]

\subsection*{A baseline: CMA-ES}

[Describe CMA-ES, describe it's drawbacks in terms of sample efficiency, and discuss its resilience against high dimensions (?)]

\subsection*{Another baseline: last-layer ensemble instead of GPs}

\section*{Vanilla Bayesian Optimization does not scale with dimensions}

[TODO: write this in more detail]

There are two reasons why Bayesian Optimization, as we discussed it in the previous post, doesn't scale well with higher dimensions.

\section*{First alternative: just drop (irrelevant) dimensions}

[Randomly dropping dimensions, or dropping dimensions according to ARD]

\section*{Additive kernels}

[Durrande's work]



\bibliographystyle{apalike}
\bibliography{biblio}

\end{document}
