\documentclass{article}
\usepackage[utf8]{inputenc}
\usepackage{fullpage}
\usepackage{fourier}
\usepackage{xcolor}
\usepackage[colorlinks=true, linkcolor=blue!75, citecolor=blue!75, urlcolor=blue!75]{hyperref}

\title{High-dimensional Bayesian Optimization: Where are we?}
\author{Miguel González Duque}
\date{}

\setlength{\parskip}{\smallskipamount}
\setlength{\parindent}{0pt}

\begin{document}

\maketitle

\section*{Introduction}

In this blogpost, I discuss some recent advancements in high dimensional Bayesian Optimization (BO), and include some practical experiments to benchmark where we currently are. The advancements I present here are definitely not all-encompassing, but rather biased towards the types of BO being done \textit{in latent spaces}, using methods like Generative Adversarial Networks, Variational Autoencoders or other types of generative models.

Some prerequisites: I assume you are familiar with Gaussian Processes and Bayesian Optimization \href{https://www.miguelgondu.com/blogposts/2023-07-31/intro-to-bo/}{to the extent I discussed them in the previous post}. Besides the benchmarks for optimization we used in said blogpost (e.g. \texttt{easom} or \texttt{cross-in-tray}), this blogpost includes benchmarks in higher-dimensional latent spaces, as well as searching for the parameters of small neural networks.

This blogpost follows, among other references listed at the end, the survey by Binois and Wycoff called \textit{A Survey on High-dimensional Gaussian Process Modeling with Application to Bayesian Optimization} \cite{BinoisWycoff:high-dimensional-bo:2022}.\footnote{\url{https://dl.acm.org/doi/pdf/10.1145/3545611}}

\section*{Vanilla Bayesian Optimization does not scale with dimensions}

[TODO: write this in more detail]

There are two reasons why Bayesian Optimization, as we discussed it in the previous post, doesn't scale well with higher dimensions.

\section*{First alternative: just drop (irrelevant) dimensions}

[Randomly dropping dimensions, or dropping dimensions according to ARD]

\section*{Additive kernels}

[Durrande's work]



\bibliographystyle{apalike}
\bibliography{biblio}

\end{document}
